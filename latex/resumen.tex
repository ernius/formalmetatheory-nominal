\documentclass{article}

    % When using XeLaTeX, the following should be used instead:
    % \documentclass[xetex, mathserif, serif]{beamer}
    %
    % The default font in XeLaTeX doesn’t have the default bullet character, so 
    % either change the font:
    % \setmainfont{XITS}
    % \setmathfont{XITS Math}
    %
    % Or change the character:
    %\setbeamertemplate{itemize items}{•}

%\usepackage[bw,references]{latex/agda}
%\usepackage[conor,references]{latex/agda}
\usepackage[hidelinks]{hyperref}
\usepackage[references,links]{agda}
\usepackage{amsmath}
\usepackage{mathtools}
\usepackage{textgreek}
\usepackage{catchfilebetweentags}
\usepackage{tipa}

\newcommand{\alp}{\ensuremath{\alpha}}
\newcommand{\lamb}[2]{\ensuremath{\AgdaInductiveConstructor{ƛ}\, #1\, #2}}
\newcommand{\inAg}{\ensuremath{\mathbin{\AgdaFunction{∈}}}}
\newcommand{\neqAg}{\ensuremath{\mathbin{\AgdaInductiveConstructor{≢}}}}
\newcommand{\ap}[2]{#1 \ensuremath{\mathbin{\AgdaInductiveConstructorFunction{·}} #2}}
\newcommand{\var}[1]{\ensuremath{\AgdaInductiveConstructorFunction{v}\, #1}}
\newcommand{\fv}{\ensuremath{{\AgdaFunction{fv}}\,}}
\newcommand{\perm}{\ensuremath{\mathbin{\AgdaFunction{∙}}}}
\newcommand{\free}{\ensuremath{\mathbin{\AgdaFunction{*}}}}
% \newcommand{\agdaf}[1]{\ensuremath{\AgdaFunction{#1}\,}}
% \newcommand{\agdaD}[1]{\ensuremath{\AgdaDatatype{#1}\,}}
% \newcommand{\agdav}[1]{\ensuremath{\AgdaBound{#1}\,}}

\DeclareUnicodeCharacter{411}{\textipa{\textcrlambda}}
\DeclareUnicodeCharacter{65288}{(}
\DeclareUnicodeCharacter{65289}{)}
\DeclareUnicodeCharacter{8336}{\ensuremath{_a}}
\DeclareUnicodeCharacter{8799}{\ensuremath{\overset{?}{=}}}
\DeclareUnicodeCharacter{8759}{\ensuremath{\dblcolon}}
\DeclareUnicodeCharacter{8718}{\ensuremath{\square}}

\title{Principles of Recursion and Induction for Nominal Lambda Calculus.}

\begin{document}

\maketitle

All the code shown is compiled in the last Agda's version 2.4.2.2 and 0.9 standard library, and can be fully accessed at:

\begin{center}
  \href{https://github.com/ernius/formalmetatheory-nominal}{https://github.com/ernius/formalmetatheory-nominal}
\end{center}

\section{Infrastructure}
\label{sec:infra}

\AgdaTarget{Λ}
\begin{figure}[!ht]
  \ExecuteMetaData[Term.tex]{term} \hspace{5px}
  \caption{Terms}
\label{fig:term}
\end{figure}

\ExecuteMetaData[Term.tex]{fresh} \hspace{5px}

\ExecuteMetaData[Atom.tex]{swap} \hspace{5px}

\ExecuteMetaData[Term.tex]{swap} \hspace{5px}

\ExecuteMetaData[Permutation.tex]{atomPermutation} \hspace{5px}

\ExecuteMetaData[Permutation.tex]{permutation} \hspace{5px}

\AgdaTarget{∼α}
\ExecuteMetaData[Alpha.tex]{alpha}


\section{Induction Principles}
\label{sec:induction}

Primitive induction over terms \AgdaDatatype{Λ} presented in figure~\ref{fig:term}:

\AgdaTarget{TermPrimInd}
\begin{figure}[!ht]
  \ExecuteMetaData[TermInduction.tex]{termPrimInduction} 
  \caption{Primitive Induction}
\label{fig:primInd}
\end{figure}

Next, we introduce another induction principle over terms with a stronger inductive hypothesis for the abstraction case, which says the property holds for all permutation of names of the body of the abstraction. This principle is proved using previous primitive induction principle.

\begin{figure}[!ht]
  \AgdaTarget{TermIndPerm}
  \ExecuteMetaData[TermInduction.tex]{termIndPermutation} 
  \caption{Permutation Induction Principle}
\label{fig:permInd}
\end{figure}

\alp-compatible predicates:

\AgdaTarget{αCompatiblePred}
\ExecuteMetaData[Alpha.tex]{alphaCompatible} \hspace{5px}

If the predicate is \alp-compatible then we can prove the following induction principle using previous  induction principle. This new principle enables us to choose the variable of the abstraction case different from a finite list of variables, in this way this principle allow us to emulate Barendregt Variable Convention (BVC).

\AgdaTarget{TermαPrimInd}
\ExecuteMetaData[TermInduction.tex]{alphaPrimInduction} \hspace{5px}

Again assuming an \alp-compatible predicate, we can prove the following principle using again the induction principle of figure~\ref{fig:permInd}:

\begin{figure}[!ht]
  \AgdaTarget{TermαIndPerm}
  \ExecuteMetaData[TermInduction.tex]{alphaIndPermutation}
  \caption{Permutation \alp Induction}
\label{fig:permAlphaInd}
\end{figure}

\section{Iteration and Recursion Principles}
\label{sec:recursion}

We want to define strong \alp-compatible functions, that is, functions over the \alp-equivalence class of terms. This functions can not depend on the abstraction variables of a term. We can resume this concept in the following definition:

\AgdaTarget{strong∼αCompatible}
\ExecuteMetaData[Alpha.tex]{strongAlphaCompatible} \hspace{5px}

We define an iteration principle over raw terms which always produces \alp-compatible functions. This is granted because abstraction variables are given by the induction principle, hidding the specific abstraction variables of the given term, in this way the result of a function defined with this iterator has no way to extract any information from abstracted variables. This principle also allow us to give a list of variables from where the abstractions variables will not to be choosen, this will be usefull to define the no capture substitution operation latter. This iteration principle is derived from the last presented induction principle in figure~\ref{fig:permAlphaInd}. 

\AgdaTarget{ΛIt}
\ExecuteMetaData[TermRecursion.tex]{termIteration} \hspace{5px}

The following result estabish the strong compatibility of previous iteration principle. This result is proved using the induction principle in figure~\ref{fig:permAlphaInd}. 

\begin{figure}[!ht]
  \AgdaTarget{lemmaΛItStrongαCompatible}
  \ExecuteMetaData[TermRecursion.tex]{iterationStrongCompatible}
  \caption{Strong \alp\ Compatibility of the Iteration Principle}
\label{fig:strongAlphaComp}
\end{figure}

From this iteration principle we directly derive the next recursion principle over terms, which also generates strong \alp-compatible functions.

\AgdaTarget{ΛRec}
\ExecuteMetaData[TermRecursion.tex]{termRecursion}

\section{Iterator Application}
\label{sec:itapp}

We present several applications of the iteration/recursive principle. In the following two sub-sections we implement two classic examples of \lamb-calculus theory. While in the appendix~\ref{sec:applications} we successfully apply our iteration/recursion principle to all the examples presented in~\cite{Norrish04recursivefunction}. This work introduces a sequence of increasing complexity functions definitions to provide a test for any principle of function definition. Each of the defined functions respects the \alp-equivalence relation, that is, are strong compatible functions by being implemented over the previously introduced itaration principle. 

\subsection{Free Variables}
\label{sec:freevar}

We implement the function that returns the free variables of a term.

\ExecuteMetaData[FreeVariables.tex]{freeVariables} \hspace{5px}

As a direct consequence of strong \alp\ compatibility of the iteration principle we obtain that \alp\ compatible terms have equal free variables. 

The relation \AgdaFunction{\_*\_} holds when a variables ocurrs free in a term.

\AgdaTarget{*}
\ExecuteMetaData[Term.tex]{free} \hspace{5px}

We can use the last induction principle (fig.~\ref{fig:permAlphaInd}) to prove the following proposition:

\AgdaTarget{Pfv*}
\ExecuteMetaData[FreeVariables.tex]{fvPred} \hspace{5px}

In the lambda abstraction obligation proof of the induction principle used, we can exclude the variable $a$\ from the abstraction variables we need to prove, simplifying in this way the required proof. We have to prove that $\forall b \neqAg a, a \inAg \fv (\lamb b M) \Rightarrow a \free \lamb b M$, knowing by inductive hypothesis that $\forall \pi , a \inAg \fv (\pi \perm M) \Rightarrow a \free (\pi \perm M)$. So $a \inAg \fv (\lamb b M)$\ and $b \neqAg a$ then we can derive that $a \inAg \fv M$ holds. Now,  instantiting the inductive hypothesis with an empty permtutation and the previous result, we have that $a \free M$,  using again that $b \neqAg a$, we can then conclude the desired result: $a \free \lamb b M$.

This flexibility comes with the extra cost that we need to prove that the predicate $\forall a , \AgdaFunction{Pfv*} a$\ is \alp-compatible in order to use this induction principle. This \alp-compatible proof is direct because \free\ is an \alp-compatible relation and the \fv function is strong \alp-compatible. So the extra cost is just the proof that \free\ is \alp-compatible.

Another approach where the last proof is automatically obtained is to define the free relation using our iteration principle and not a data type as previously done.

\AgdaTarget{free}
\ExecuteMetaData[FreeVariables.tex]{free} \hspace{5px}

For the variable case we return the type of the propositional equality, inhabited only if the searched variable is equal to the term variable. The application case is the disjoint union of the types returned by the recursive calls, that is, the union of the variable free ocurrence evidence in the applied terms. Finally, in the abstraction case we can choose the abstraction variable to be different from the searchead one, so we can ignore the abstraction variable, and return just the recursive call, containing the evidence of any variable free ocurrence in the abstraction body. 

This free predicete impementation is strong compatible by construction because we build it from our iterator principle, so given any variable $a$\ and two \alp-compatible terms $M,N$, the returned set should be the same. So is direct that if the predicate holds for $M$ (which means that the returned set is inhabited for $M$), then the predicate should also hold for $N$. 

From this point we can  do an analog proof of \AgdaFunction{Pfv*} proposition, but now using this new free predicate definition which is \alp-compatible by construction. This give us a framework where we can define strong compatible functions and also \alp-compatible predicates over terms, and then prove properties about theses functions and predicates using our induction principle that provides us with the BVC.

\subsection{Substitution}
\label{subst}

We implement the no capture substitution operation, we avoid any variable capture giving as variables to not to choose from as variable abtractions: the substituted variable and the free variables of the replaced term.

\ExecuteMetaData[Substitution.tex]{substitution} \hspace{5px}

Again because of the strong \alp-compability of the iteration principle we obtain the following result for free:

\AgdaTarget{lemmaSubst1}
\ExecuteMetaData[Substitution.tex]{lemmaSubst1} \hspace{5px}

Using the induction principle in figure~\ref{fig:permInd} we prove:

\AgdaTarget{lemmaSubst2}
\ExecuteMetaData[Substitution.tex]{lemmaSubst2} \hspace{5px}

Finally, from the two previous result we directly obtain next substitution lemma.

\AgdaTarget{lemmaSubst}
\ExecuteMetaData[Substitution.tex]{lemmaSubst} \hspace{5px}

Remarkably all this result are directly derived from the first primitive induction principle, and no need of induction on the length of terms or accesible predicates were needed in all of this formalization.

\appendix

\section{Iteration/Recursion Applications}
\label{sec:applications}

In the following sections we successfully apply our iteration/recursion principle to all the examples from~\cite{Norrish04recursivefunction}. This work presents a sequence of increasing complexity functions definitions to provide a test for any principle of function definition, where each of the given functions respects the \alp-equivalence relation.

\subsection{Case Analysis and Examining Constructor Arguments}
\label{sec:caseanalysis}

The following family of functions distinguishes between constructors returning the constructor components, giving in a sense a kind of \emph{pattern-matching}.

\begin{minipage}{.5\textwidth}
\[\begin{array}{rll}
isVar &: \Lambda \rightarrow& Maybe\ ( Variable ) \\
isVar &(v\ x)         &= Just \\
isVar &(M \cdot N)   &= Nothing \\
isVar &(\lambda x M) &= Nothing
\end{array}\]
\end{minipage}
\begin{minipage}{.5\textwidth}
\[\begin{array}{rll}
isApp &: \Lambda \rightarrow& Maybe\ (\Lambda \times \Lambda) \\
isApp &(v\ x)          &= Nothing \\
isApp &(M \cdot N)   &= Just (M , N) \\
isApp &(\lambda x M) &= Nothing
\end{array}\]
\end{minipage}

\centerline{
\begin{minipage}{.5\textwidth}
\[
\begin{array}{rll}
isAbs &: \Lambda \rightarrow& Maybe\ (Variable \times \Lambda) \\
isAbs &(v\ x)         &= Nothing \\
isAbs &(M \cdot N)   &= Nothing \\
isAbs &(\lambda x M) &= Just (x , M)
\end{array} \]
\end{minipage}}

Next we present the corresponding codifications using our iteration/recursion principle:

\AgdaTarget{isVar} \AgdaTarget{isApp} \AgdaTarget{isAbs}
\ExecuteMetaData[Norrish.tex]{constructors} \hspace{5px}

\subsection{Simple recursion}
\label{sec:rec}

The size function returns a numeric measurement of the size of a term.

\centerline{
\begin{minipage}{.5\textwidth}
\[
\begin{array}{rll}
size &: \Lambda \rightarrow& \mathbb{N} \\
size &(v\ x)         &= 1 \\
size &(M \cdot N)   &= size (M) + size (N) + 1 \\
size &(\lambda x M) &= size (M) + 1 
\end{array} \]
\end{minipage}}

\AgdaTarget{size}
\ExecuteMetaData[Norrish.tex]{size} \hspace{5px}

\subsection{Alfa Equality}

Next functions decides the \alp-equality relation between two terms.

\AgdaTarget{equal}
\ExecuteMetaData[Norrish.tex]{alphaEqual} \hspace{5px}

Observe that \AgdaFunction{isAbs} function also normalises N, so it is correct in last line to ask if the two binders are equal.

\subsection{Recursion Mentioning a Bound Variable}
\label{sec:recbound}

The $enf$ function is true of a term if it is in $\eta$-normal form, the $fv$ function returns the set of a term’s free variables and was previously defined.

\centerline{
\begin{minipage}{.8\textwidth}
\[
\begin{array}{rll}
enf &: \Lambda \rightarrow& Bool \\
enf &(v\ x)         &= True \\
enf &(M \cdot N)   &= enf (M) \wedge enf (N) + 1 \\
enf &(\lambda x M) &= enf (M) \wedge (∃ N, x / isApp (M) == Just (N , v\ x) \Rightarrow x \in fv (N))
\end{array} \]
\end{minipage}}

\hspace{5px}

\AgdaTarget{enf}
\ExecuteMetaData[Norrish.tex]{enf} \hspace{5px}

\subsection{Recursion with an Additional Parameter}

Given the ternary type of possible directions to follow when passing through a term $({Lt, Rt, In})$, corresponding to the two sub-terms of an application constructor and the body of an abstraction, return the set of paths (lists of directions) to the occurrences of the given free variable in a term, where $cons$\ insert an element in front of a list.

\[
\begin{array}{rrll}
&vposns &: Variable \times \Lambda \rightarrow & List\ (List\ Direction) \\
&vposns &(x , v\ y)         &= if\ (x == y)\ then\ [[]]\ else\ []   \\
&vposns &(x , M \cdot N)    &= map\ (cons\ Lt)\ (vposns\ x\ M) +\!\!+  \\
& & &\ \ \ map\ (cons\ Rt)\ (vposns\ x\ N) \\
x \neq y \Rightarrow&vposns &(x ,\lambda y M)   &= map\ (cons\ In)\ (vposns\ x\ M) \\
\end{array} \]

Notice how the condition guard of the abstraction case is translated to the list of variables from where not to chose from the abstraction variable.

\AgdaTarget{vposns}
\ExecuteMetaData[Norrish.tex]{vposns} \hspace{5px}

\subsection{Recursion with Varying Parameters and Terms as Range}

A variant of the substitution function, which substitutes a term for a variable, but further adjusts the term being substituted by wrapping it in one application of the variable named "0" per traversed binder.

\[
\begin{array}{rrll}
&sub' &: \Lambda \times Variable \times  \Lambda &   \rightarrow  \Lambda  \\
&sub' &(P , x , v\ y)         &= if\ (x == y)\ then\ P\ else\ (v\ y)   \\
&sub' &(P , x , M \cdot N)    &= (sub' (P , x , M)) \cdot (sub' (P , x , N)) \\
\left. 
\begin{array}{c}
y \neq x  \ \wedge \\
 y \neq 0\ \  \wedge \\
 y \not\in fv(P) \\
\end{array} \right\} \Rightarrow&sub' &(P , x , \lambda y M)   &= \lambda y (sub' ((v\ 0) \cdot M , x , M)) \\
\end{array} \]


To codify with our iterator principle this function we must change the parameters order, so our iterator principle now returns a function that is waiting the term to be substituted. In this way we manage to vary the parameter through the iteration.

\AgdaTarget{sub'}
\ExecuteMetaData[Norrish.tex]{sub} \hspace{5px}

\end{document}